\documentclass[usenatbib]{mnras}
\usepackage[T1]{fontenc}
\usepackage{ae,aecompl}

\usepackage{graphicx}	% Including figure files
\usepackage{amsmath}	% Advanced maths commands
\usepackage{amssymb}	% Extra maths symbols
\usepackage{array}

\usepackage{multirow}
\usepackage{multicol}
\usepackage{blindtext}
\newcolumntype{?}{!{\vrule width 1pt}}
\newcommand{\Msun}{\,{\rm M}$_{\odot}$\,}
\newcommand{\Mpch}{\,{\rm Mpc}\,\ifmmode h^{-1}\else $h^{-1}$\fi}
\newcommand{\kpch}{\,{\rm kpc}\,\ifmmode h^{-1}\else $h^{-1}$\fi}
\newcommand{\kpc}{\,{\rm kpc}\,}
\newcommand{\kms}{\,{\rm km}\ s$^{-1}$\,}

%%%%%%%%%%%%%%%%%%% TITLE PAGE %%%%%%%%%%%%%%%%%%%
\title[Superclusters in velocity divergence fields]{Superclusters in velocity divergence fields}
\author[Pe\~naranda-Rivera et al.]{
\parbox[t]{\textwidth}{
    {J. D. Pe\~naranda-Rivera $^1$,} 
    {D. L. Paipa-Le\'on$^{1}$,}
    {S. D. Hern\'andez-Charpak$^{1,2}$,}\\
    {J. E. Forero-Romero $^{1}$}
}
\\\\
$^{1}$ Departamento de F\'isica, Universidad de los Andes, Cra. 1
  No. 18A-10 Edificio Ip, CP 111711, Bogot\'a, Colombia \\
$^{2}$ Center for Neuroprosthetics and Brain Mind Institute, Swiss
  Federal Institute of Technology (EPFL), CH-1015, Lausanne,
  Switzerland\\  
}

% These dates will be filled out by the publisher
\date{Accepted XXX. Received YYY; in original form ZZZ}

% Enter the current year, for the copyright statements etc.
\pubyear{2019}


% Don't change these lines
\begin{document}
\label{firstpage}
\pagerange{\pageref{firstpage}--\pageref{lastpage}}
\maketitle

%%%%%====MYMARK========
\maketitle
\begin{abstract}
Galaxy superclusters can be defined as regions of converging galaxy
velocity flows. Such structures have quantifiable properties such as
mass, volume and shape. The most precise quantification of such an
object has been done on our local supercluster, Laniakea. In this work
the properties of Laniakea are compared to the expectations from a
$\Lambda$CDM Universe. Cosmological N-body simulations are used in
order to find cosmic flow fields of dark matter halos and find
superclusters using a watershed algorithm. In addition, the
distributions of volume of the superclusters found in those
simulations are quantified and compared with known values for
Laniakea. The latter process was done for both Planck 2015 cosmologies
and cosmologies with extreme values of $\Omega_m$ and $\sigma_8$
parameters. In all those cases Laniakea is found on the right side of
the distribution, placing it as an atypical object in a cosmological
context.  
\end{abstract}

\begin{keywords}
%%PREGUNTAR
%galaxies:supercluster --- simulation:cosmological --- filter:gaussian --- methods:numerical
\end{keywords}


%=========================================================================
%		PAPER CONTENT
%=========================================================================

%*************************************************************************

\section{Introduction}
% La introduccion y el abstract es lo ultimo que se escribe.


In 2014 astronomers used a peculiar velocities map  to 
define the Laniakea supercluster of galaxies
\citep{2014Natur.513...71T}.  
Their main input was the Cosmicflows-2 catalog of distances and radial
velocities that provide them with coverage of distances up to 400 Mpc
[CITA A COSMICFLOWS-2]. 
By applying a Wiener filter \citep{Zaroubi_1999} they obtained the
reconstruction of the 3-dimensional velocity field. 
From this reconstructions they our local galaxy supercluster, Laniakea.
Laniakea thus represents the region of inflowing peculiar velocities
that include the Milky Way. 
The volume spanned by Laniakea is close to isotropic with an 
approximate diameter of $160$ Mpc encompassing a mass of
$\approx 10^{17}$ \Msun.

From a general computational point of view the definition of Laniakea poses
an interesting problem, namely defining regions of converging velocity
flows.
This conceptual statement has been recently exploited by \cite{Dupuy_2019} 
to use streamlines found in peculiar velocity data and find
superclusters. 
They make use of the concept of basins of attraction/repulsion, defined as
regions where velocity flows are observed to be inflowing/outflowing
as consequence of gravitational forces. 
In practice, they track streamlines in space to define gravitational
basins of attraction/repulsion.   

\cite{Dupuy_2020} studied in detail the influence of the details in
the algorithm parameters in the detected gravitational basins. 
The computation of the streamlines involves the application of a
fourth order Runge-Kutta method. 
In this way, obtaining structures using this method is influenced by
the choice of the integration parameters. 
The interpretation of basins of attraction or repulsion
also depends on the direction in which the integration is carried
out. 

Additionally, the
influence of a smoothing parameter (named smoothing scale) is also a
focus of interest in the latter work, and the results found are
expected to be the same in this paper. 

In this paper we are interested in superclusters, which are defined as
the union of the elements of the cosmic network that form an
identifiable structure. 
Studying this type of structures involves the
same difficulties of studying the cosmic web as they are not found by
a unique method. In particular, luminosity density data has been one
of the selected ways to find and classify superclusters in
observational data, as presented in \cite{Lietzen_2016} and
\cite{Bagchi_2017}. In addition, numerical simulations represent a
powerful tool in the study of superclusters as they allow investigators to
test different cosmological parameters and also to consider the time
evolution of the cosmic web components
\citep{Einasto_2019,einasto2020evolution}. Our main interest remains
in a kinematic approach to this problem, which will let us introduce a
supercluster finding algorithm that uses the divergence of the
velocity field of DM halos. 



The research developed by \cite{2014Natur.513...71T} corresponds to a kinematic approach to the problem of defining superclusters in observational data. 






In the same way, a joint work between the
observations and the simulations has been carried out in order to
fully understand this pattern in the universe. We can find an example
of the observational approach in \cite{Kim_2016}, where the
filamentary structures surrounding the Virgo cluster are analysed
using data from the HyperLeda database. 
Other examples of this
approach can be found in \cite{Santiago_Bautista_2020},
\cite{van_de_Weygaert_2014} and \cite{Lares_2017}. Finally, a
computational approach is presented in
\cite{10.1111/j.1365-2966.2009.14885.x} where a dynamical
classification of the cosmic web is applied to cosmological N-body
simulations.  

We aim to quantify if Laniakea can be considered as a typical or atypical 
structure in our current cosmological context. In order to achieve
this objective, we will apply a watershed algorithm
\citep{BeucherWatershed1979} which runs over the divergence field of
the velocity and segments the simulations used into isolated
groups. The process of structure finding using watershed methods is
explored in \cite{10.1111/j.1365-2966.2007.12125.x} where a extensive
analysis is made to connect image processing with astrophysics. In
particular, we will implement a watershed by immersion algorithm,
which depends only on the values of the divergence field.  

The following discussion presents a general approach to the
methodology to be used in the work, as well as the results found. In
this way, section 2 presents the numerical tools used, including the
operation of the finder algorithm, and section 3 presents the results
found after applying our method to different simulations. 


\section{Numerical Setup}
\label{sec:numerical_setup}
\subsection{Cosmological N-body data}


In this paper we use simulations from the Abacus Cosmos project.
We use public data corresponding to dark matter only N-body
simulations executed on a cube of side length $720$\ \Mpch with
$1440^3$ particles.  
A set of these simulations assume a $\Lambda$CDM Planck 2015 cosmology
with different realizations for the initial conditions. 
Another set uses the same phases for the initial conditions, but
different cosmological parameters generated by a Latin hypercube
centered around the Planck 2015 parameters.  
The resolution of these simulations correspond to a DM particle mass
of $\sim 1 \times 10^{10}$ \Msun.
We use the public Friend-of-Friends catalogs at redshift $z=0.1$.
The halos included in those catalogs have a lower mass bound of 
maximum circular velocity of $V_{\rm circ}=50$\,\kms.
We trim these catalogs to halos with maximum circular velocity larger
than $300$ \kms. 
We have checked that changing changing this threshold to $200$ \kms does not
significantly impact our results. 



\subsection{Velocity interpolation, smoothing and divergence calculation}  

We interpolate the peculiar velocities for all halos on a grid using a
Cloud-In-Cell (CIC) scheme. 
We use a cubic grid with $360^3$ voxels, which means that each voxel
is a cube of $2$ \Mpch on a side.  
This resolution is deliberately smaller than the nominal resolution
used in the Wiener filter reconstruction performed on the
CosmicFlows-2 data used to define Laniakea. 
After the interpolation we smooth the velocity field using a gaussian
filter of physical scale $\sigma_s$.  
We compute the velocity divergence on this smooth velocity field. 
The purpose of this smoothing is to mimic the process of Wiener filter
reconstruction, which provides the equivalent of a smooth peculiar
velocity field. 
In explore different smoothing scales, $\sigma_s$/\Mpch, of $2$, $6$,
$10$, $14$ and $20$. 



\begin{figure*}
    \centering
    \includegraphics[width=240pt]{plot_power_spectrum_field_rs.pdf}
    \includegraphics[width=240pt]{plot_power_spectrum_tracers_rs.pdf}
    \caption{Power spectrum}
    \label{fig:power_spectrum}
\end{figure*}

\begin{figure*}
    \centering
    \includegraphics[width=240pt]{plot_corrfunc_field_rs.pdf}
    \includegraphics[width=240pt]{plot_corrfunc_tracers_rs.pdf}
    \caption{Autocorrelation function}
    \label{fig:power_spectrum}
\end{figure*}

\subsection{Watershed algorithm: Supercluster-Finder}

We use a watershed algorithm \citep{BeucherWatershed1979} on the velocity divergence field to find the
superclusters.
The algorithm segments the whole volume by assigning each voxel to a unique supercluster. 
Our implementation works as follows. 
We sweep the cells in the divergence grid from the lowest values to the highest, i.e. from
the high density regions with converging flows to the lowest density regions with divergent
flows.
For the $i$-th cell under consideration we check whether its 26 neighbors have already been assigned to a group. 
If all the neighbors are unassigned, then this $i$-th cell starts a
new group; if the majority of already assigned cells belongs to the
$n$-th group, then this cell belongs to that group.
In case of draw between groups, the cell is assigned to the supercluster with the lowest $n$ value.
At the end of the sweep all cells have been assigned to a group. 
In all our calculations we take periodical boundary conditions into
account.  
We highlight that once the divergence field is provided, this algorithm
does not have free parameters based either on divergence, density or distance thresholds. 
Finally, each group found by the algorithm is interpreted as a supercluster 
with a volume computed by counting the number of voxels belonging to it.

\subsection{The effective smoothing scale in Cosmicflows-2 velocity reconstrution}

The results of the watershed algorithm are strongly dependent on the
smoothing scale used to estimate the compute the velocity field.
We need to have an estimate of the effective smoothing scale in the observations in order to compare our results against Laniakea.

\cite{2015MNRAS.452.1052L} used the velocity reconstruction on cosmicflows-2 data, the same reconstruction used to define Laniake.
The paper includes a computation of the dimensionless shear tensor $\Sigma_{\alpha\beta}=-\frac{1}{2H_0}\left(\frac{\partial v_\alpha}{\partial r_\beta}+\frac{\partial v_\beta}{\partial v_\alpha}\right)$ where $\alpha$ and $\beta$ represent the three cartesian components of the velocity $v$ and the position $r$ and $H_0$ is the Hubble-Lema\^itre constant.
With this definition the trace of the shear tensor is equal to minus the divergence of the velocity field.


The authors say that the formal reconstruction for the Wiener Filter velocity reconstruction is $5$\Mpch and the derivatives are computed over scales of $2.5$\Mpch. 
We use these two values as a bracket to be used later at the moment of comparing our results against observations.

A second piece of evidence that help us to pin down the effective smoothing scale comes from their reports on the numerical values of the shear tensor eigenvalues.
The authors found that the trace of the dimensionless shear tensor at the Local Group and Cen A location to be $0.030$ and $0.015$, respectively. 
These values translate into $3.0$ and $1.5$ in units of km $h$ Mpc$^{-1}$ s$^{-1}$. 
In Section XX we show that the divergence distributions in our calculations with a similar order of magnitude values correspond to 
to smoothing scales smaller than $6$\Mpch.



\section{Results}






\begin{figure}
    \centering
    \includegraphics[width=240pt]{correlation_length.pdf}
    \caption{Correlation length $R_{\delta\delta}$ as a function of the truncation scale $R_s$.}
    \label{fig:correlation length}
\end{figure}



\begin{figure*}
    \centering
    \includegraphics[width=212pt]{smooth_watershed_01.pdf}
    \includegraphics[width=211pt]{smooth_watershed_05.pdf}
    \caption{Figures on top show a random (200\,\Mpch)$^2$ cut on the
      X axis of the divergence grid for two different values of
      $\sigma_s$. Figures on the bottom show the classification made
      by the Watershed method for the two regions shown on top and
      where different superclusters have different colors. These two
      charts also contain a centered circle with the same diameter as
      Laniakea. It can be seen that Laniakea is much bigger than any
      other supercluster found using both $\sigma_s$.} 
    \label{fig:1Pert}
\end{figure*}



This section presents in the first place the influence of $\sigma_s$
on the velocity fields, its divergence and the resulting supercluster
segmentation. 
In a second stance we show the influence of cosmic variance and
different cosmological parameters on the volume distribution function
for superclusters. 

\subsection{Smoothing scale influence on the velocity and divergence fields}
\label{VDF effects}

Figure \ref{fig:smooth_vel_dist} shows the velocity norm distribution
for four different values of $\sigma_s$. 
As $\sigma_s$ increases the distribution becomes more symmetric with a
decreasing width, as expected.
The same trend is observed in Figure \ref{fig:smooth_grad_dist} for
the divergence of the velocity field.

This is neatly summarized in  Figure \ref{fig:std_smooth}.
The standard deviation of the divergence has a strong dependence on
the smoothing length.
The standard deviation goes from values of $3$  km s$^{-1}$ Mpc$^{-1}$ $h$ 
for a smoothing scale of $2$ \Mpch down to $0.075$ km s$^{-1}$
Mpc$^{-1}$ $h$ for a smoothing scale of $20$ \Mpch.
That means that increasing the smoothing scale by a factor of $8$
reduces the width of the divergence distribution by a factor of
$40$. 

\subsection{Smoothing scale influence on the resulting superclusters}
\label{sec:superclusters_influence}

The strong effects of the smoothing scale on the velocity and its
divergence is immediatly imprinted onto the results of the watershed
algorithm.


Finally, the influence of $\sigma_s$ on the sighting of structures
formation must be analyzed. This effects can be seen in Figure
\ref{fig:1Pert} where the same cross-section of the divergence field
grid is shown but smoothed with $\sigma_s$ = 2\,\Mpch and $\sigma_s$ =
10\,\Mpch (top diagrams). In addition, this chart presents the effects
of this parameter on the groups selected by watershed algorithm
(bottom diagrams), but this is discussed in the next section. These
plots show a random section of the divergence grid with size
(200\,\Mpch)$^2$. With small $\sigma_{s}$ many structures are visible
in a fine and defined network in the divergence grid, while large
$\sigma_{s}$ differentiates big structures of high accretion,
i.e. high negative, with barely any resolution of smaller
filaments. Also, using large $\sigma_{s}$ causes peak values of
divergence to be smoothed and values of the field move closer to  the
average value of the field.  








\subsubsection{Interpretation of low divergence regions}
Once the Gaussian process is realized in the simulations, several
regions of negative divergence are identified. These regions are
interpreted as places in space where matter is coming together at high
ratios. As a result, after the interpolation and smoothing are
executed, negative divergence regions are expected to have the largest
values of absolute velocities. These high negative divergence regions
can be observed in the divergence field chart using $\sigma_s =
10$\,\Mpch in figure \ref{fig:1Pert}. Notice how in these structures
the divergence is much more negative in the center of the
agglomerations than in the rest of the structure. The latter
corresponds to a structure formation process, interpreted in this
context as the region where a supercluster could possibly be, and will
be the first regions considered once we apply the watershed
algorithm. 


\subsection{$\sigma_s$ influence on the Watershed algorithm}


In the previous section we described how varying $\sigma_s$ has a
significant impact on the values of the velocity and divergence
fields. Since Watershed algorithm takes the divergence field and
segments it into disjoint groups (superclusters), it is expected that
$\sigma_s$ also modifies the result of the algorithm by
transitivity. In addition, we mentioned that increasing the Gaussian
length causes peak values of negative divergence to become large
structures with negative divergence. Therefore, since the algorithm
classifies according to the behaviour of these structures, if they get
bigger, superclusters detected by the algorithm will become greater
too. As a result, a drop on the number of superclusters detected must
be observed. The behaviour of the number of superclusters found as a
function of $\sigma_s$ is shown in figure \ref{fig:Nclusters}. Indeed,
we observe a considerable reduction of the number of superclusters
found when $\sigma_s$ is raised. In addition figures \ref{fig:1Pert}
and \ref{fig:HISTVMD2} shows the effects of changing the value of the
Gaussian length on the volumes of the superclusters detected. 

As mentioned in section \ref{VDF effects} figure \ref{fig:1Pert} shows
a random section of size (200\,\Mpch)$^2$ of the divergence grid using
$\sigma_s$ = 2\,\Mpch and $\sigma_s$ = 10\,\Mpch. Moreover, this
figure also presents the groups found by the algorithm corresponding
to this section, with a circle of the same diameter as Laniakea in the
middle of each chart. First, it can be seen that there is a clear
correspondence between the state of the divergence grid and the
structures found. If space is dominated by null divergence values and
small regions with negative divergence($\sigma_s = 2\,\Mpch$), small
groups will be detected. On the other hand, if the size of these
negative divergence regions is increased, large groups will be
detected by the algorithm. However, even when the value of $\sigma_s$
is raised from 2\,\Mpch to 10\,\Mpch, superclusters detected by
Watershed method are much smaller than the reported size of Laniakea
by \cite{2014Natur.513...71T}. The behaviour of the volume
distribution of the superclusters for different $\sigma_s$ values is
presented in figure \ref{fig:HISTVMD2}, where the same simulation is
smoothed using $\sigma_s$ = 2\,\Mpch, 6\,\Mpch, 14\,\Mpch and
20\,\Mpch. In addition, a vertical line is drawn to mark the reference
of the volume of Laniakea. For low $\sigma_s$ values, i.e. 2\,\Mpch
and 6\,\Mpch, no supercluster detected by the algorithm is as large as
Laniakea. In both cases Laniakea is found at the right side of the
distribution and is almost one or two orders of magnitude bigger than
the largest structures found for 2\,\Mpch and 6\,\Mpch
respectively. On the other hand, once we take values of $\sigma_s$
higher than 10\,\Mpch, Laniakea starts to be inside of the
distribution, as in $\sigma_s$ = 14\,\Mpch. Finally, we observe that
its volume becomes the must common one found by the algorithm when
$\sigma_s$ = 20\,\Mpch is taken. This result gives us a first hint of
the relation of Laniakea with the simulated superclusters found by
Watershed method. It seems to be that Laniakea is an atypical and very
large supercluster. We will delve into this topic in the following
section. 



\begin{figure}
    \centering
    \includegraphics[width=240pt]{num_superclusters.pdf}
    \caption{Behaviour of the number of superclusters found in a
      simulation as a function of the smoothing scale parameter
      $\sigma_{Vox}$. Effectively a smaller $\sigma_{Vox}$ implies
      recognizing many more structures. 
This behavior was expected since each structure carries a smaller
volume and a larger number of them is required to fill the entire
space of the simulation.}  
    \label{fig:Nclusters}
\end{figure}


\begin{figure*}
    \centering
    \includegraphics[width=345pt]{vol_different_sigmas.pdf}
    \caption{On this plot the volume of Laniakea is compared with the
      volumes of superclusters found using the watershed algorithm
      with different values of $\sigma_s$. A vertical line is showing
      the correspondent volume of Laniakea is also shown. The same
      AbacusCosmos\_720box\_planck simulation was smoothed using
      $\sigma_s$ = 2\,\Mpch, 6\,\Mpch, 14\,\Mpch and 20\,\Mpch and the
      volume distribution of the superclusters found is presented in
      this figure.}  
    \label{fig:HISTVMD2}
\end{figure*}




\subsection{Laniakea classification}

At this point, we have already seen that the volume distribution of the selected superclusters has a significant dependence on the choice of $\sigma_s$. Therefore, if we want to compare the classification made by the algorithm with reported volume measures of Laniakea, we must choose wisely the value of the Gaussian length. Thus, the value reported by \cite{2014Natur.513...71T} in his studies on CosmicFlows-2 data is used. From now on, the value of the Gaussian length is fixed and equals 10\Mpch. 

Now, in order to complete the analysis of Laniakea, it is important to
establish a pattern by using different sets of simulations. We are
interested in the impact that different cosmological parameters may
have in the volume distribution of superclusters, as well as the
behaviour of this distribution when different time evolution are
considered. 


\subsubsection{Planck 2015 parameters}

To begin with, a set of 5 simulations of the
Abacus\_Cosmos\_720box\_planck catalog is considered. This set of
simulations fulfills two main functions. First, it allows us to
observe the effects of the time evolution on the resulting volume
distribution. Second, as the simulations are executed using Planck
2015 cosmological parameters, we are able to establish a reference
point with generally accepted cosmological parameters. In this way,
figure \ref{fig:planck} shows the volume distributions of the
superclusters found by Watershed algorithm using this data set. This
chart also shows a vertical line where the reported volume of Laniakea
is located. As expected from figure \ref{fig:1Pert}, Laniakea is
larger than any other simulation found by the algorithm, and it does
not change with different time evolution. For instance, the volume of
our local supercluster is found on the right side and out of the
distributions, which implies that there is no structure as big as
Laniakea. Consequently, we conclude that Laniakea is an atypical
supercluster if a normal cosmology is considered.  



However, the atypical behaviour found using Planck 2015 parameters may
change if extreme cosmologies are used in this process. The effect of
the mass density of the universe ($\Omega_m$), as well as the value of
the matter fluctuations in present time ($\sigma_8$) might cause
volume distributions to be affected and, therefore, it can affect how
typical or atypical Laniakea is. Thus it is important to consider
different simulations that satisfy this extreme conditions. 


\subsubsection{Extreme values of $\Omega_m$ and $\sigma_8$}

Now, a set of 4 simulations of the Abacus\_Cosmos\_720box catalog as
well as an Abacus\_Cosmos\_720box\_planck reference simulation is
studied. From Abacus\_Cosmos\_720box we extract the 4 simulations that
correspond to the highest and lowest values of $\Omega_m$ and
$\sigma_8$ found in this catalog. Thereby, the selected set contains
the simulations with the highest value of $\sigma_8$ and $\Omega_m$
(0.999 and 0.367 respectively) and the lowest value of the same
parameters (0.647 and 0.253 respectively). The resulting volume
distributions for this data set are shown in figure
\ref{fig:diferentes}. A vertical line representing the volume of
Laniakea is shown too. We observe that even when extreme cosmologies
are considered, the volume of Laniakea is located outside the
distributions and, therefore, this supercluster conserves its atypical
size and behaviour. 



\section{Conclusions}
\label{sec:conclusions}


Throughout this investigation the main objective was to find a way to classify the Laniakea structure using N-body simulations. Thus, we made use of the dark matter halos velocity field divergence for the classification of structures in space. In addition, in order to achieve a robust result, different cosmologies were used to obtain the final results of the project.


First, a clear influence of the $\sigma_s$ parameter was observed when performing the segmentation process of the spatial grid through the Watershed algorithm. The latter behavior is produced by a great variation on the divergence field once this parameter is changed, as seen in figures \ref{fig:smooth_grad_dist} and \ref{fig:std_smooth}. In particular, the higher the value of $\sigma_s$, the larger the size of the convergent galaxy flow structures found in the simulations. As a consequence, when applying the algorithm with a greater $\sigma_s$ it is found that the superclusters are larger (behavior presented in figure \ref{fig:1Pert}).

However, the correct value of $\sigma_s$ corresponds to a physical choice. In particular, a value approximately of 10\Mpch. Therefore, once this parameter is adjusted, our algorithm has no free parameters that affect the generated distribution. Moreover, in order to classify the structure of Laniakea, we decided to use two types of cosmologies. 

The first set of cosmologies used corresponds to a generally accepted set of cosmological parameters given by Planck 2015 study. The volume distributions generated using this set of simulations, shown in figure \ref{fig:planck}, gives us a first idea that Laniakea is an atypical supercluster in terms of its size. On the other hand, a set of simulations with different cosmological parameters was used to evaluate their impact on the distributions. In particular, simulations using extreme values of $\Omega_m$ and $\sigma_s$ were selected. The resulting volume distributions are presented in figure \ref{fig:diferentes}. The same behavior as in Planck 2015 simulations is observed. 

In conclusion, we point out that the algorithm used in effect allows an effective partition of the simulations using the divergence field. Furthermore, it worked properly to allow us classifying Laniakea as an atypical supercluster in a cosmological context.




\bibliographystyle{mnras}
\bibliography{references}



\end{document}
